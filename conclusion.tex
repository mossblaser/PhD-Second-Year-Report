\chapter{Conclusion}
	
	% Brain is interesting but tough to deal with, existing machines struggle to
	% make the grade in terms of performance.
	%
	% Neuromorphic hardware offers the potential to scale up much more easily.
	% SpiNNaker is a fun example of a digital approach to this. It avoids problems
	% faced by existing systems.
	%
	% TODO: Finish-off.
	
	The brain is one of the least understood entities in science and despite vast
	interest and intense research still largely remains a mystery. Models of brain
	function have grown increasingly sophisticated both in their detail and sheer
	scale.
	
	Though conventional supercomputers have continued to answer to the needs of
	certain classes of neural simulations, large scale models based on simple
	neurons have spawned a range of special purpose simulation platforms to
	support their needs. These `neuromorphic' systems aim to provide a balance of
	computational power and communication capacity which better suit large scale
	spiking neural networks. Neuromorphic devices typically implement specialised
	network interconnect which supports efficient multicast communication of
	extremely small messages.
	
	SpiNNaker is a neuromorphic system which combines large numbers of small,
	mobile-phone grade computer processors with a specially designed
	interconnection network. In small scale, single-board SpiNNaker prototypes,
	all communication between chips in the system used a low-power asynchronous
	protocol. As the system has been scaled up to multiple boards, new
	interconnect based on HSS links has been introduced to make interconnecting
	boards in large systems practical.
	
	Preliminary work has explored the capabilities of HSS link technology, using
	SpiNNaker as a case study. The results of this exploration include a tool and
	methodology for designing the wiring required for large SpiNNaker systems
	while adhering to constraints on wire length imposed by HSS technology. As a
	result of this work, a novel method of augmenting SpiNNaker's board-to-board
	links to form a small-world network was presented. This inexpensive
	augmentation has the potential both to improve the throughput and latency of
	the SpiNNaker interconnect and save power.
	
	To support this work, an interconnection network simulator for SpiNNaker has
	been built and verified within which the benefits and effects of small-world
	connectivity may be evaluated. In addition, work has commenced to build the
	foundations for implementing such extensions to SpiNNaker's HSS link hardware.
	
	Upcoming work will aim to realise and evaluate extensions to SpiNNaker's HSS
	links. The evaluation of this work will be based on a suite of both `real' and
	synthetic large-scale neural models using the NEF.  The initial construction
	of this suite of benchmarks will, in itself, offer a contribution to the
	understanding of the network behaviour of neural networks within SpiNNaker.
	
	It is planned that a small-world networking scheme will be developed based on
	the results of simulation work. Since one of the aims of this work is to
	reduce load in the system, a complementary line of research is also planned to
	develop methods for reducing power consumption due to underutilised HSS links.
	
	It is hoped that this research will both advance the abilities of the existing
	SpiNNaker platform while also paving the way towards improving the way HSS
	links are used within neural simulators. With future versions of SpiNNaker
	being likely to make use of HSS at every level of its network, its effective
	use may be guided by this work.

