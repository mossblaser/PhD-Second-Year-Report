\begin{tikzpicture}[ thick
                   , blockstyle/.style={ draw
                                       , rectangle
                                       , minimum height=2em
                                       }
                   ]
	\pgfmathsetlengthmacro{\bitlen}{0.5em}
	\pgfmathsetlengthmacro{\bitdashlen}{0.3em}
	
	% #1: Node to draw on
	% #2: Number of bits
	\newcommand{\bitdashes}[2]{
		\pgfmathsetmacro{\boundaries}{#2-1}
		\begin{scope}[help lines]
			\foreach \x in {1,...,\boundaries}{
				\draw ([shift={(\x*\bitlen,0)}] #1.north west)
				   -- ++(0, -\bitdashlen);
				
				\draw ([shift={(\x*\bitlen,0)}] #1.south west)
				   -- ++(0, \bitdashlen);
			}
		\end{scope}
		
		\draw [ decorate
		      , decoration={ brace
		                   , amplitude=1ex
		                   , raise=0.2ex
		                   }
		      ]
		      (#1.south east)
		   -- node [anchor=north, yshift=-1.2ex] {#2 bits}
		      (#1.south west);
	}
	
	\node (control)
	      [ blockstyle
	      , minimum width=8*\bitlen
	      ]
	      {control};
	\bitdashes{control}{8}
	
	\node (key)
	      [ blockstyle
	      , minimum width=32*\bitlen
	      , right=-\the\pgflinewidth of control
	      ]
	      {key};
	\bitdashes{key}{32}
	
	\node (payload)
	      [ blockstyle
	      , minimum width=32*\bitlen
	      , right=-\the\pgflinewidth of key
	      ]
	      {optional payload};
	\bitdashes{payload}{32}
	
\end{tikzpicture}

