\chapter{Background}
	
	\section{Simulating brains}
		
		% What is done, for what purpose
		
		\subsection{History of artificial neural networks}
			
			% Three generations and their differing aims.
		
		\subsection{Computation}
			
			% Number of neurons and synapses, amount of state, differential equation
			% model, discrete time, typically 0.1-1.0 ms time-steps.
		
		\subsection{Communication}
			
			% Number of spikes produced, the form that spikes take, multicast,
			% possible changes due to learning. Especially fun for real-time.
	
	
	\section{Supercomputer technology}
		
		% Outline the top of the line from top500.
		
		\subsection{Anatomy}
			
			% What is in a typical super computer. Lots of CPU/GPU/Accel nodes with
			% comparatively limited interconnect. Vast amounts spent on power, cooling
			% etc.
		
		\subsection{Interconnect}
			
			% What these machines are designed to carry between nodes. Typically
			% packet switched: messages containing destination and payload, hopping
			% from node to node in the system.
			
			\subsubsection{Topology}
				
				% Topology has influence on structure of the system, super computers are
				% optimised for local access of data near by. Two topologies are
				% popular: trees and tori.
				%
				% Trees have low hop counts and expensive routers, tori have higher hop
				% counts and cheaper routers. Both are easily partitioned.
			
			\subsubsection{Routing}
				
				% Route large (kb long) packets, mainly point-to-point, guaranteed
				% delivery. Lots of buffering. Generally tightly integrated with link
				% technology.
			
			\subsubsection{Link technology}
				
				% Universally high-speed-serial over optical and electrical links.
				% Optical is expensive but good for long distances. Always-on and so the
				% pressure is on to keep links fully loaded.
	
	
	\section{Neuromorphic computing}
		
		% Hardware which mimics biological systems more directly, essentially
		% optimised computers for neural simulations.
		
		\subsection{Analogue and mixed-mode}
			
			% Using analogue rather than digital electronics to accelerate the
			% computation of the various functions involved. Analogue circuits can
			% directly implement the differential equations required, can be very
			% tricky to calibrate, also inherently fixed. Interconnects often tend to
			% be digital to simplify things.
			
			\subsubsection{BrainScaleS}
				
				% Whole silicon wafer: extremely fast, extremely low power. Rather hard
				% to calibrate. Fair amount of flexibility in neuron model.
				%
				% Two interconnection systems: L1 and L2. L1 is circuit switched mesh
				% for low-power, low-latency connections, but very inflexible. L2 uses
				% FPGAs to translate into 10 gigabit Ethernet or other links. More
				% flexible and allows some scaling. Spikes multicast.
				%
				% Difficult to scale.
			
			\subsubsection{Neurogrid}
				
				% Very simple neural model, many chips in a tree. Communications are
				% digital and certain neuron features, such as connection weights are
				% modelled using probabilistic delivery. Spikes are serialised for each
				% chip. Between chips a tree is used.
				%
				% TODO: Does it use multicast?
				% TODO: Find out more about interconnect here.
			
		\subsection{Digital}
			
			% Use digital implementations of neurons, typically optimised in some
			% fashion.
			
			\subsubsection{Bluehive}
				
				% FPGA based, custom high-speed-serial interconnect, same link tech as
				% super computers, 3D torus network. Spike duplicated for all
				% destinations.
			
			\subsubsection{SpiNNaker}
				
				% CPU based, custom asynchronous multicast interconnect but scaling up
				% with high-speed serial 2D hexagonal torus network.
	
	
	\section{SpiNNaker network architecture}
		
		% Greater detail intro to SpiNNaker since it will be the focus of this work
		%
		% Overview of system: cores, chips, boards, racks, cabinets. Network is
		% hexagonal torus with nodes being chips.
		
		\subsection{Routing \& multicast}
			
			% Packet types and sizes. Table based routing, generally, multicast
			% routing in SpiNNaker. Also describe the sort of fun which can be had
			% with multicast, mention power savings due to less hops. Mention router
			% simplicity, limitations, assumptions.
		
		\subsection{Link technologies}
			
			% Between and within chips use asynchronous links. Between boards this
			% would be impractical due to number of wires and performance over long
			% wires. Instead, high-speed serial via FPGAs is used. Minimal torus
			% construction.
	
	\section{High-Speed Serial (HSS)}
		
		% Technology now used by most super computers and high-speed comms tasks.
		% More complex than parallel but they scale up throughput better.
		
		\subsection{Eliminating skew}
			
			% Parallel vs Serial, problems with parallel.
		
		\subsection{Clock recovery}
			
			% Requires frequent transitions (e.g. using 8b/10b) even on idle line,
			% makes idling expensive.
		
		\subsection{Clock correction}
			
			% Requires buffering clock/correction to deal with clock differences.
		
		\subsection{Error recovery}
			
			% Must retransmit packets with errors. Requires buffers.
		
		\subsection{Flow control}
			
			% Now unlimited amounts of data can be sent so must have way of applying
			% back-pressure. Again, means more buffering and complexity.
