\chapter{Preliminary work}
	\label{sec:preliminary-work}
	
	% TODO: Add more linking together perhaps?
	
	The preliminary work outlined in this chapter initially concentrates on
	developing a fuller understanding of the characteristics of the
	interconnection networks used in SpiNNaker. This work has developed into the
	foundations for new extensions to the SpiNNaker network which will form the
	main contributions of this project.
	
	The chapter begins with the development of an efficient wiring scheme for
	large SpiNNaker systems which accounts for the properties of the interconnect
	technology. This is followed by a discussion of an extension to the network
	which exploits the suggested wiring scheme to inexpensively transform the
	network into a small world network with improved latency.  As part of an
	effort to validate potential changes to the network, a model of SpiNNaker's
	interconnection network has been built. A brief description of this work is
	included along with details of collaborative work to develop new modelling
	techniques.  The chapter concludes with an outline of extensions being made to
	the existing HSS board-to-board links within SpiNNaker.
	
	\section{SpiNNaker machine wiring}
		
		% Constraint of electrical high-speed-serial is that long wires are not
		% allowed. For tori, this is a problem due to wrap-around.
		
		\subsection{Reducing wiring length}
			
			% Description of the folding technique
		
		\subsection{`SpiNNer' Wiring Guide Generator}
			
			% Python Library and various front-ends have been produced which produce
			% wiring diagrams and listings. The tool has been successfuly used to
			% assemble machines (pictures!) and provide orders for cabling for the
			% next machines.
	
	
	\section{Small-world networks}
		
		% Small world networks are: ... Allow lots of short paths, this means low
		% latency, low power and less contention. We describe ways of augmenting the
		% SpiNNaker network, making use of spare connections and resource for
		% board-to-board links while taking advantage of the physical arrangement
		% described in the previous section.
		
		\subsection{Constructing small-world networks}
			
			% Add random links and lo-and-behold, smallworldness!
		
		\subsection{Modelling and results}
			
			% Don't need too many links to get a good return in terms of reduced
			% path length. Limiting results to short wires doesn't hurt too much,
			% especially after folding.
		
		\subsection{Further work}
			
			% Non-uniformity to be studied. Also, plan to model this with more
			% realistic traffic and experiment on SpiNNaker.
	
	
	\section{SpiNNaker network modelling}
		
		% Colaboration with others researching simulator platforms, due for journal
		% submission in coming months, final stages of writing. Context of the work:
		% want to try out modelling on different platforms. My contribution: some
		% writing and the development of an accurate software simulator for
		% SpiNNaker's interconnect. This work indirectly offers interesting insights
		% into behaviours due to the board-to-board links.
		
		\subsection{Software simulator model}
			
			% Model 18 cores as traffic generators/consumers. Links are modelled as
			% delay elements (accurately captures req/ack links), internal arbitration
			% scheme is modelled along with pipelined router.
		
		\subsection{Results}
			
			% Experiments performed showed highly matched behaviour between the model
			% and SpiNNaker but reveal the effects of the non-homogeneity introduced
			% by board-to-board links for certain traffic patterns.
		
		\subsection{Further work}
			
			% Plans to include models of SpiNNaker links. Preliminary trials at
			% accurately modelling these have been unsuccessful. Will be used to guide
			% work on these links.
	
	
	\section{SpiNNaker FPGA connectivity}
		
		% Work started at CapoCaccia with input from various parties to get
		% SpiNNaker peripherals working. The start of a project to get peripherals
		% properly integrated and generally add some routing smarts to the FPGAs
			
			\subsection{Existing infrastructure}
				
				% Description of what exists already in the board-to-board links,
				% outline the protocol's principles. Also describe the rdy/vld
				% interface.
			
			\subsection{Extension peripherals}
				
				% Describe principle of how a router will be integrated and streams
				% merged.
			
			\subsection{Proof-of-concept `spiking button' peripheral}
				
				% Describe the demo and hardware used
				
				\subsubsection{Handling multiple link speeds}
					
					% Streams of packets being produced at different speeds due to slower
					% peripheral FPGA board. Describe the interface's role, outline the
					% mechanism and avoidance of buffering.
				
				\subsubsection{Merging packet streams}
					
					% Again, describe the problem, and the challenges of doing this
					% without losing throughput.
			
			\subsection{Further work}
				
				% Next steps will involve getting basic routing up and running, then on
				% to my planned project, see next chapter.
	
