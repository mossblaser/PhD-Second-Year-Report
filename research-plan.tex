\chapter{Research plan}
	
	% Generally planning to massage the existing system to make better use of
	% serial links, looking at powering down strategies and testing things both in
	% simulation (using existing simulator) and also using real traffic in
	% SpiNNaker. To this end I will be continuing background activities involving
	% working with neural modelling.
	%
	% Gantt chart here...
	
	\section{FPGA-based routing}
		
		% Develop routing scheme for SpiNNaker packets suitable for peripheral, host
		% and preliminary small-world communications via the SpiNNaker links. This
		% scheme will continue to make use of the current HSS protocol blocks.
		
		\subsection{Host communication}
			
			% Snooping, loading, etc.
		
		\subsection{Peripheral communication}
			
			% Configuration and hetrogenaity
		
		\subsection{Board-to-board communication}
			
			% Small world networks
	
	
	\section{SpiNNaker traffic analysis}
		
		% To better target efforts and gain further understanding, work with Mundrew
		% on Nengo network experiments to assess performance.  Must try to get a
		% picture of what demand for links will be like. High-speed links offer a
		% unique peek into network traffic beyond what is offered by existing
		% counters.
	
	\section{High-speed serial power control}
		
		% Work to try and make some power savings by powering down idle links or
		% reducing speed (to reduce link transitions), taking into account latency
		% concerns for SpiNNaker.
		
		\subsection{Power-down and bring-up}
			
			% Specify latency issues, describe current technologies for doing this and
			% appropriate signalling stuff. See PCI-E, S-ATA and friends.
		
		\subsection{Speed changing}
			
			% Possible application of link speeds being set in response to demand.
		
		\subsection{Always-on links}
			
			% Use a selection of always-on serial links when demand is low or others
			% are being powered up. How do small-world links fit in this picture?
		
		\subsection{`Santos-28' test chip}
			
			% A 28 nm test chip is being developed with TU Dresden to evaluate various
			% technologies' performance for inclusion in a second generation SpiNNaker
			% chip. There may be room in this space for my own involvement in this
			% work at short notice.
	
	
	\section{Performance evaluation}
		
		% Can build on work by Evangelos to see how performance in terms of power
		% works out with various FPGA based things going on. Hopefuly steal some of
		% his test networks. Can also sneak myself into work on SPAUN performance
		% messings.
