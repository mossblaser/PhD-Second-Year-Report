\chapter{Research plan}
	\label{sec:research-plan}
	
	% Generally planning to massage the existing system to make better use of
	% serial links, looking at powering down strategies and testing things both in
	% simulation (using existing simulator) and also using real traffic in
	% SpiNNaker. To this end I will be continuing background activities involving
	% working with neural modelling.
	
	\begin{figure}
		\center
		\begin{tikzpicture}[thick,x=0.25cm]

%%%%%%%%%%%%%%%%%%%%%%%%%%%%%%%%%%%%%%%%%%%%%%%%%%%%%%%%%%%%%%%%%%%%%%%%%%%%%%%%
% Hacked-up Gantt Library
%%%%%%%%%%%%%%%%%%%%%%%%%%%%%%%%%%%%%%%%%%%%%%%%%%%%%%%%%%%%%%%%%%%%%%%%%%%%%%%%

% An entry in the Gantt chart. Takes a label, start offset, length and slack.
% Also defines a pair of labels "[label] start" and "[label] end" which can be
% used for drawing dependency lines.
\newcommand{\ganttEntry}[4]{
	% Label
	\node (label)
				[below=1.5ex of label.south east,anchor=east,minimum height=1.7em]
				{#1}
				;
	\coordinate (gantt labels end) at (label.south east);
	
	% Box
	\draw [fill=white]
	      ([shift={(#2   ,-.9ex)}]label.north east) rectangle
	      ([shift={(#2+#3,0.9ex)}]label.south east);
	
	% The tips of the box
	\coordinate (#1 end)
	         at ([shift={(#2+#3,0.9ex)}]label.south east);
	\coordinate (#1 start)
	         at ([shift={(#2   ,-.9ex)}]label.north east);
	
	% Slack line
	\draw [ultra thick]
	      ([shift={(#2+#3,0)}]$(label.north east)!0.5!(label.south east)$)
	   -- ++(#4,0);
}

\newcommand{\ganttDep}[2]{
	\draw [->,red] (#1 end) -| (#2 start);
}

\newcommand{\ganttVSep}[2]{
	\draw [#2] ([shift={(#1,0)}]gantt labels start) -- ([shift={(#1,0)}]gantt labels end);
}

% A new gantt chart. Takes a list of x-offset/label/major-label tuples. For each
% tuple a line is created with x-offset from the previous line and the span is
% labelled with "label". If major-label given, a major label will be drawn
% centered over the previous entries up until the last major-label.
\newenvironment{gantt}[1]{
	% Start the list of labels
	\node (label) [white] {Ag};
	\coordinate (gantt labels start) at (label.north east);
	\def\periods{#1}
}{
	\begin{scope}[on background layer]
		% Thick line separating from labels
		\draw (gantt labels start) -- (gantt labels end);
		
		% Start of the area covered by a "major" label
		\coordinate (gantt maj label start) at (gantt labels start);
		
		\foreach \x/\lab/\mlab in \periods {
			\coordinate (next gantt labels start) at ([shift={(\x,0)}]gantt labels start);
			\coordinate (next gantt labels end)   at ([shift={(\x,0)}]gantt labels end);
			
			% Minor label
			\node at ($(gantt labels start) !0.5! (next gantt labels start)$)
			      [anchor=west,rotate=90]
			      {\lab}
			      ;
			
			% Separator
			\ifthenelse{\equal{\mlab}{}}{
				\draw [help lines] (next gantt labels start) -- (next gantt labels end);
			}{
				\draw [help lines,thick] (next gantt labels start) -- (next gantt labels end);
			}
			
			\coordinate (gantt labels start) at (next gantt labels start);
			\coordinate (gantt labels end)   at (next gantt labels end);
			
			% Major label
			\ifthenelse{\equal{\mlab}{}}{}{
				\coordinate (next gantt maj label start) at (gantt labels start);
				\node at ($(gantt maj label start) !0.5! (next gantt maj label start)$)
				      [yshift=1cm,anchor=south]
				      {\mlab}
				      ;
				\coordinate (gantt maj label start) at (next gantt maj label start);
			}
		}
	\end{scope}
}



%%%%%%%%%%%%%%%%%%%%%%%%%%%%%%%%%%%%%%%%%%%%%%%%%%%%%%%%%%%%%%%%%%%%%%%%%%%%%%%%
% The Chart...
%%%%%%%%%%%%%%%%%%%%%%%%%%%%%%%%%%%%%%%%%%%%%%%%%%%%%%%%%%%%%%%%%%%%%%%%%%%%%%%%

\begin{gantt}{
	4/Sep/, 4/Oct/, 4/Nov/, 4/Dec/2014,%
	3/Q1/, 3/Q2/, 3/Q3/, 3/Q4/2015,%
	3/Q1/, 3/Q2/2016%
}
	\ganttEntry{Nengo Network Experiments}       {0}{4}{8}
	\ganttEntry{Small-World Modelling}           {0}{4}{2}
	\ganttEntry{Small-World HSS Impelementation} {4}{8}{4}
	\ganttEntry{Small-World Benchmarking}        {10}{6}{1}
	\ganttEntry{USB 3.0 Interface}               {16}{2}{2}
	\ganttEntry{HSS Power Management}            {16}{5}{2}
	\ganttEntry{Power Management Benchmarking}   {21}{3}{1}
	
	\ganttEntry{Thesis Writing}     {24}{8}{2}
	
	\ganttDep{Small-World Modelling}{Small-World HSS Impelementation}
	\ganttDep{Nengo Network Experiments}{Small-World Benchmarking}
	\ganttDep{Small-World HSS Impelementation}{HSS Power Management}
	\ganttDep{HSS Power Management}{Power Management Benchmarking}
\end{gantt}

\end{tikzpicture}

		\caption[Gantt chart overview of research plan.]{Gantt chart overview of
		research plan. Boxes indicate expected duration of a task, thick lines
		indicate slack and red arrows show dependencies between tasks. Note the
		non-linear scale.}
		\label{fig:plan-gantt}
	\end{figure}
	
	The preliminary work so-far has focused on 
	
	\section{FPGA-based routing}
		
		% Develop routing scheme for SpiNNaker packets suitable for peripheral, host
		% and preliminary small-world communications via the SpiNNaker links. This
		% scheme will continue to make use of the current HSS protocol blocks.
		
		\subsection{Host communication}
			
			% Snooping, loading, etc.
		
		\subsection{Peripheral communication}
			
			% Configuration and hetrogenaity
		
		\subsection{Board-to-board communication}
			
			% Small world networks
	
	
	\section{SpiNNaker traffic analysis}
		
		% To better target efforts and gain further understanding, work with Mundrew
		% on Nengo network experiments to assess performance.  Must try to get a
		% picture of what demand for links will be like. High-speed links offer a
		% unique peek into network traffic beyond what is offered by existing
		% counters.
	
	\section{High-speed serial power control}
		
		% Work to try and make some power savings by powering down idle links or
		% reducing speed (to reduce link transitions), taking into account latency
		% concerns for SpiNNaker.
		
		\subsection{Power-down and bring-up}
			
			% Specify latency issues, describe current technologies for doing this and
			% appropriate signalling stuff. See PCI-E, S-ATA and friends.
		
		\subsection{Speed changing}
			
			% Possible application of link speeds being set in response to demand.
		
		\subsection{Always-on links}
			
			% Use a selection of always-on serial links when demand is low or others
			% are being powered up. How do small-world links fit in this picture?
		
		\subsection{`Santos-28' test chip}
			
			% A 28 nm test chip is being developed with TU Dresden to evaluate various
			% technologies' performance for inclusion in a second generation SpiNNaker
			% chip. There may be room in this space for my own involvement in this
			% work at short notice.
	
	
	\section{Performance evaluation}
		
		% Can build on work by Evangelos to see how performance in terms of power
		% works out with various FPGA based things going on. Hopefuly steal some of
		% his test networks. Can also sneak myself into work on SPAUN performance
		% messings.
