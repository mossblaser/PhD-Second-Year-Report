\chapter{Research plan}
	\label{sec:research-plan}
	
	% Generally planning to massage the existing system to make better use of
	% serial links, looking at powering down strategies and testing things both in
	% simulation (using existing simulator) and also using real traffic in
	% SpiNNaker. To this end I will be continuing background activities involving
	% working with neural modelling.
	
	\begin{figure}[b!]
		\center
		\begin{tikzpicture}[thick,x=0.25cm]

%%%%%%%%%%%%%%%%%%%%%%%%%%%%%%%%%%%%%%%%%%%%%%%%%%%%%%%%%%%%%%%%%%%%%%%%%%%%%%%%
% Hacked-up Gantt Library
%%%%%%%%%%%%%%%%%%%%%%%%%%%%%%%%%%%%%%%%%%%%%%%%%%%%%%%%%%%%%%%%%%%%%%%%%%%%%%%%

% An entry in the Gantt chart. Takes a label, start offset, length and slack.
% Also defines a pair of labels "[label] start" and "[label] end" which can be
% used for drawing dependency lines.
\newcommand{\ganttEntry}[4]{
	% Label
	\node (label)
				[below=1.5ex of label.south east,anchor=east,minimum height=1.7em]
				{#1}
				;
	\coordinate (gantt labels end) at (label.south east);
	
	% Box
	\draw [fill=white]
	      ([shift={(#2   ,-.9ex)}]label.north east) rectangle
	      ([shift={(#2+#3,0.9ex)}]label.south east);
	
	% The tips of the box
	\coordinate (#1 end)
	         at ([shift={(#2+#3,0.9ex)}]label.south east);
	\coordinate (#1 start)
	         at ([shift={(#2   ,-.9ex)}]label.north east);
	
	% Slack line
	\draw [ultra thick]
	      ([shift={(#2+#3,0)}]$(label.north east)!0.5!(label.south east)$)
	   -- ++(#4,0);
}

\newcommand{\ganttDep}[2]{
	\draw [->,red] (#1 end) -| (#2 start);
}

\newcommand{\ganttVSep}[2]{
	\draw [#2] ([shift={(#1,0)}]gantt labels start) -- ([shift={(#1,0)}]gantt labels end);
}

% A new gantt chart. Takes a list of x-offset/label/major-label tuples. For each
% tuple a line is created with x-offset from the previous line and the span is
% labelled with "label". If major-label given, a major label will be drawn
% centered over the previous entries up until the last major-label.
\newenvironment{gantt}[1]{
	% Start the list of labels
	\node (label) [white] {Ag};
	\coordinate (gantt labels start) at (label.north east);
	\def\periods{#1}
}{
	\begin{scope}[on background layer]
		% Thick line separating from labels
		\draw (gantt labels start) -- (gantt labels end);
		
		% Start of the area covered by a "major" label
		\coordinate (gantt maj label start) at (gantt labels start);
		
		\foreach \x/\lab/\mlab in \periods {
			\coordinate (next gantt labels start) at ([shift={(\x,0)}]gantt labels start);
			\coordinate (next gantt labels end)   at ([shift={(\x,0)}]gantt labels end);
			
			% Minor label
			\node at ($(gantt labels start) !0.5! (next gantt labels start)$)
			      [anchor=west,rotate=90]
			      {\lab}
			      ;
			
			% Separator
			\ifthenelse{\equal{\mlab}{}}{
				\draw [help lines] (next gantt labels start) -- (next gantt labels end);
			}{
				\draw [help lines,thick] (next gantt labels start) -- (next gantt labels end);
			}
			
			\coordinate (gantt labels start) at (next gantt labels start);
			\coordinate (gantt labels end)   at (next gantt labels end);
			
			% Major label
			\ifthenelse{\equal{\mlab}{}}{}{
				\coordinate (next gantt maj label start) at (gantt labels start);
				\node at ($(gantt maj label start) !0.5! (next gantt maj label start)$)
				      [yshift=1cm,anchor=south]
				      {\mlab}
				      ;
				\coordinate (gantt maj label start) at (next gantt maj label start);
			}
		}
	\end{scope}
}



%%%%%%%%%%%%%%%%%%%%%%%%%%%%%%%%%%%%%%%%%%%%%%%%%%%%%%%%%%%%%%%%%%%%%%%%%%%%%%%%
% The Chart...
%%%%%%%%%%%%%%%%%%%%%%%%%%%%%%%%%%%%%%%%%%%%%%%%%%%%%%%%%%%%%%%%%%%%%%%%%%%%%%%%

\begin{gantt}{
	4/Sep/, 4/Oct/, 4/Nov/, 4/Dec/2014,%
	3/Q1/, 3/Q2/, 3/Q3/, 3/Q4/2015,%
	3/Q1/, 3/Q2/2016%
}
	\ganttEntry{Nengo Network Experiments}       {0}{4}{8}
	\ganttEntry{Small-World Modelling}           {0}{4}{2}
	\ganttEntry{Small-World HSS Impelementation} {4}{8}{4}
	\ganttEntry{Small-World Benchmarking}        {10}{6}{1}
	\ganttEntry{USB 3.0 Interface}               {16}{2}{2}
	\ganttEntry{HSS Power Management}            {16}{5}{2}
	\ganttEntry{Power Management Benchmarking}   {21}{3}{1}
	
	\ganttEntry{Thesis Writing}     {24}{8}{2}
	
	\ganttDep{Small-World Modelling}{Small-World HSS Impelementation}
	\ganttDep{Nengo Network Experiments}{Small-World Benchmarking}
	\ganttDep{Small-World HSS Impelementation}{HSS Power Management}
	\ganttDep{HSS Power Management}{Power Management Benchmarking}
\end{gantt}

\end{tikzpicture}

		\caption[Gantt chart overview of research plan.]{Gantt chart overview of
		research plan. Boxes indicate expected duration of a task, thick lines
		indicate slack and red arrows show dependencies between tasks. Note the
		non-linear scale.}
		\label{fig:plan-gantt}
	\end{figure}
	
	The preliminary work so-far has focused on the analysis and exploration of
	extensions to SpiNNaker's interconnection network. In particular, it focuses
	on the high-speed serial links which connect chips on different boards within
	the system. The proposed research plan is summarised in figure
	\ref{fig:plan-gantt}.
	
	The proposed work can be divided into two distinct parts. The first
	investigates the use of small-world, or small-world-like networks within
	actual SpiNNaker systems. The second extends this to attempt to improve power
	efficiency within SpiNNaker by powering-off or slowing down links where
	lightly-loaded alternatives exist.
	
	To measure success, realistic network traffic from neural simulations, along
	with the collection of suitable metrics will be required. Building a
	representative selection of benchmark networks, along with the infrastructure
	for assessing their impact on SpiNNaker's interconnect will be a key part of
	the work undertaken.
	
	This chapter will outline each of the key research aims and the work proposed
	in approximately chronological order.
	
	\section{Nengo network experiments}
		
		Nengo and the SPA provide an actively used neural modelling architecture
		capable of generating large scale neural models. Unlike many other
		approaches Nengo is already being used to construct large models from simple
		neurons, the design target of SpiNNaker. The SPAUN model, for example,
		contains 2.5 million simple neurons \cite{eliasmith12}, while models
		produced by students at the two-week Nengo 2014 Summer School constructed
		networks containing almost half a million neurons, even within this short
		timespan. As well as being large, Nengo models generally feature easily
		verified high level-behaviour. This feature enables easier identification of
		system failures and can help guide efforts to debug issues.
		
		In addition to the above, Nengo's SpiNNaker implementation lends itself to
		convenient analysis due to its unconventional implementation. Though still
		simulating spiking neurons, the spikes themselves are not transmitted over
		SpiNNaker's network, instead, the value represented by the spikes is
		transmitted. This approach exploits the observation that typically a
		population with hundreds of neurons will represent only a vector of one or
		two values. Though the value must be transmitted every time step (typically
		1ms) and requires a `long' 72 bit SpiNNaker packet, a large population of
		neurons in Nengo typically produces a large number of spikes each time step,
		each requiring a `short' 40 bit packet. For example, for a population of
		1,000 neurons representing a two-valued vector and firing at an average rate
		of 10 Hz there would be around ten 40 bit packets (400 bits) transmitted
		each millisecond.  Alternatively, if the represented value was transmitted,
		only two 72 bit packets would be required (144 bits). In reality, many Nengo
		models feature firing rates far higher than the 10 Hz average found in the
		brain and which is the design target of SpiNNaker. As a result, this
		mechanism can result in significant savings while also making it practical
		to run models on SpiNNaker.
		
		Transmission of represented value has the side-effect of making the rate of
		packet transmission deterministic: the same number of values must be
		transmitted every millisecond, regardless of the simulated network's state.
		This means that artificially reproducing the traffic patterns and modelling
		them analytically is greatly simplified.
		
		Initially, work will be carried out to analyse the effects of different
		traffic patterns produced by large Nengo networks in SpiNNaker. This is
		planned to consist of the creation of a new `neuron type' which instead of
		actually modelling a neuron simply generates an equivalent traffic pattern
		and records network behaviour. Since most Nengo networks use a single neuron
		type, this means a single parameter change is required for any given
		network. Additionally, this approach allows great flexibility in the metrics
		recorded since cores will be largely idle.
		
		In addition to `real' Nengo models a number of synthetic models designed to
		stress network resources will also be created.
		
		This work is expected to produce two major outcomes: an improvement in
		Nengo's performance on SpiNNaker and also the production of a suite of
		benchmarks based on real Nengo networks. The former improvement is expected
		to result primarily from a better understanding of the effect that different
		patterns of packet transmission have on network performance. Some
		improvement is also anticipated through improvements to the placement and
		routing algorithms used for allocating simulation components to cores in the
		system. This work is planned for publication though a venue has not yet been
		selected.
		
	
	\section{Small-World Networks for SpiNNaker}
		
		% Develop routing scheme for SpiNNaker packets suitable for peripheral, host
		% and preliminary small-world communications via the SpiNNaker links. This
		% scheme will continue to make use of the current HSS protocol blocks.
		
		In order to exploit the benefits of small-world network topologies work will
		be carried out to attempt to implement them within SpiNNaker's
		board-to-board interconnection. Work will begin in tandem with Nengo network
		experiments to extend the Tickysim network simulator with small world
		connectivity. Guided by these simulation results, this will be followed by
		an implementation in SpiNNaker. This implementation will then be evaluated
		using the Nengo benchmarks above.
		
		There are three main challenges associated with this task which are outlined
		in this section. The first is the construction of a routing scheme which is
		easily implemented within the available FPGA logic. The second is the method
		by which random links are added and which effect this has on physical
		assembly difficulties and performance. The last, which will be undertaken
		only if time allows, is the possibility of routing packets between FPGAs on
		the same board to allow, for example, packets to `hop over' an entire board.
		
		\subsection{Routing}
			
			One of the principle challenges of implementing new network topologies
			within the FPGAs on a SpiNNaker board is that all routing must take place
			within the FPGA itself. Within a SpiNNaker chip, routing is handled by
			dedicated routing hardware featuring a large associative memory able to
			search a table of up to 1,024 routing entries in a single cycle. Within
			the FPGA, such structures are very expensive. Additionally, each FPGA is
			responsible for sixteen individual streams of traffic, all of which must
			be routed without reducing their throughput.
			
			In preliminary work, a very simple single-entry key and mask routing
			scheme proved adequate for host and potentially peripheral connectivity.
			This scheme consists of a mask and comparison which can be implemented
			very cheaply within the FPGA logic.  This approach is adequate here
			because generally there will be only a few I/O devices in a system
			compared with the number of neurons and so it is feasible to allocate an
			entire bit (or more) of the system's routing key space to indicate such
			packets.  For small world networks, however, allocating a part of the key
			space to indicate when each small-world link should be used would be
			prohibitive due to the number of such links.
			
			One possible approach to this problem is to exploit the relative speed
			differences between the FPGA logic and the links to SpiNNaker chips. A
			SpiNNaker chip link can nominally transmit a (short) packet in around 167
			$\mu$s while FPGA logic clocked at 75 MHz cycles in around 13 ns. This
			means that if a router with a throughput of one packet per cycle was
			implemented, up to twelve links could be handled simultaneously without
			impacting on throughput. This means that two such routers would be
			required to handle each FPGA's sixteen links. A feasibility study will be
			required to establish the potential capability of such routers given the
			resources available on the FPGAs on each board.
			
			An alternative approach is to allow the FPGAs to modify the routing keys
			of packets which pass through them. If it is found that most packets will
			only use a single small-world link, a single bit could be defined within
			the key space to indicate that a packet should use the first small-world
			link it encounters. This bit would be cleared as it crosses the small
			world link after which the packet would be routed normally. This method
			would be able to exploit the existing key and mask routing scheme already
			implemented at the expense of greater complexity in the placement and
			routing system.
			
			The selection of the approach to be used will depend on the results of
			Tickysim simulations. Additionally, the choice may be guided by time
			constraints since a basic implementation of the second approach is likely
			to be significantly less time consuming than a full FPGA based router
			implementation.
		
		\subsection{Random Connectivity}
			
			The method by which random connectivity is introduced into a small-world
			network can have a significant impact on both the amount of manual labour
			required to install it and also the actual performance gains achieved. One
			approach is to generate random connections and produce a list of wiring
			instructions for a technician to assemble. The alternative is to allow the
			technician to manually add the connections at random.
			
			Computer generated random connections have the advantage that they can be
			both unbiased and also may be post processed to remove links which, for
			example, happen to connect two already neighbouring or nearly-neighbouring
			boards. The main downside of this approach is that the manual effort
			required to manually assemble such connections may be great, especially
			for large machines where there may be as many as 3,600 sockets where
			connections may potentially be made.
			
			The alternative approach, while straightforward for a technician may be
			marred by the poor quality of human generated randomness
			\cite{figurska08}. Additionally, humans are unlikely to be able to spot
			and avoid `obviously poor' choices due to the complexity of the wiring
			schemes derived in preliminary work.
			
			Some work is required in this area to establish the efficacy of either
			approach. Simple experiments building on the simple graph based models
			used in preliminary work will be carried out to attempt to judge the
			importance of the random decision making.
		
		\subsection{Board-Hopping}
			
			% TODO: Diagram?
			
			The FPGAs on a SpiNNaker boards are interconnected allowing them to
			communicate locally via a currently unused HSS ring network. If time
			allows, this could be used to directly forward packets which would
			otherwise simply pass directly through all the chips the board before
			emerging on the other side. This mechanism may potentially reduce
			contention for link resources on intervening chips. This mechanism is a
			variant of an express cube \cite{dally91} which requires no additional
			implementation resources since it exploits existing hierarchy in the
			network.
		
		\subsection{Evaluation}
			
			The suite of Nengo benchmark networks described in the previous section
			will be used to evaluate the performance of the system. It is hypothesised
			that large, heavily connected networks will benefit from the additional
			connectivity since these are where longer connections are likely to exist
			as well as greater contention for link resources.
	
	
	\section{USB 3.0 Interface}
		
		As part of the collaboration to develop a high speed interface for host
		connectivity with SpiNNaker an FPGA design will be produced upon the
		completion of a prototype interface board. No firm date has been given for
		arrival of a prototype board and so the dates indicated in figure
		\ref{fig:plan-gantt} are approximate. This work will consist of an extension
		of the UART proof-of-concept design which interfaces with a Cypress FX-3 USB
		3.0 interface chip in place of the FTDI UART-over-USB interface chip.
		
		Given the timely availability of this device, it will allow superior
		monitoring of the SpiNNaker network since packets travelling over a link can
		be `snooped' via the USB link without impacting the system's performance.
		Additionally, it would allow software models of high-bandwidth I/O devices
		and host communications to be implemented.
	
	
	\section{High-speed serial power management}
		
		% Work to try and make some power savings by powering down idle links or
		% reducing speed (to reduce link transitions), taking into account latency
		% concerns for SpiNNaker.
		
		Power consumption is an important consideration for large SpiNNaker
		machines. At present, SpiNNaker's high-speed serial links operate at full
		speed at all times, even when no traffic is present, potentially wasting a
		significant amount of energy. While numerous techniques for power managing
		HSS technologies exist, these tend to be optimised toward more conventional
		super computer loads.
		
		In general, power savings are achieved either by powering down idle links
		completely or by running links at low speeds. These techniques introduce
		problems for latency-sensitive, largely lightly-loaded interconnection
		networks as in SpiNNaker. Typically power savings are achieved in HSS links
		by powering off a link or operating it at lower speed. This process
		introduces an inherent delay since it requires the sender and receiver's
		PLLs to re-lock on to the new frequency which can take on the order of 1 ms
		\cite{xilinx14}. Though this latency may be acceptable for coarse-grained
		power management, it would prohibit the SpiNNaker realtime goal of packets
		being delivered within 1 ms.
		
		This section describes two complementary proposals for making use of these
		existing low-power techniques while also maintaining realtime performance.
		The section concludes with a possible collaboration with the Technical
		University of Dresden where a link with better performance may be developed.
		
		\subsection{Link oversampling}
			
			% During PLL locking, could still have good performance
			
			During PLL acquisition, the clock signal produced is of low quality and
			gradually approaches the ideal waveform. One design hypothesis is that
			during this acquisition period, data may still be sent at low speeds, for
			example by duplicating every bit sent across the link (oversampling).
			Additionally, further error detection information may be included during
			this time to handle the increased error rate. This technique may reduce
			the latency between link speed changes and useful data propagating across
			a link.
			
			A large part of this work will consist of characterising the quality of
			the link during PLL acquisition and thus determining the oversampling
			required. This technique will also require reimplementation of parts of
			the HSS hard IP to be reimplemented in FPGA logic.
		
		\subsection{Traffic redirection}
			
			% Send traffic around other edges of a triangle
			
			\begin{figure}
				\center
				\begin{tikzpicture}
	
	\clip (1cm,1.25cm) rectangle +(4cm,3.5cm);
	
	\begin{scope}[hexagonXYZ,minimum size=0.25cm,inner sep=0]
		\foreach \y in {1,...,8}{
			\foreach \x in {1,...,8}{
				\ifthenelse{\equal{\x\y}{43} \OR \equal{\x\y}{54} \OR \equal{\x\y}{53}}{
					\node (chip \x \y) at (\x,\y) [fill,circle] {};
				}{
					\node (chip \x \y) at (\x,\y) [fill=gray,circle] {};
				}
			}
		}
	\end{scope}
	
	\begin{scope}[help lines]
		\foreach \y in {2,...,7}{
			\foreach \x in {2,...,7}{
				\pgfmathtruncatemacro{\xx}{\x+1}
				\pgfmathtruncatemacro{\yy}{\y+1}
				\draw (chip \x \y) -- (chip \xx \yy);
				
				\pgfmathtruncatemacro{\xx}{\x-1}
				\pgfmathtruncatemacro{\yy}{\y-1}
				\draw (chip \x \y) -- (chip \xx \yy);
				
				\pgfmathtruncatemacro{\xx}{\x+1}
				\pgfmathtruncatemacro{\yy}{\y+0}
				\draw (chip \x \y) -- (chip \xx \yy);
				
				\pgfmathtruncatemacro{\xx}{\x-1}
				\pgfmathtruncatemacro{\yy}{\y+0}
				\draw (chip \x \y) -- (chip \xx \yy);
				
				\pgfmathtruncatemacro{\xx}{\x+0}
				\pgfmathtruncatemacro{\yy}{\y+1}
				\draw (chip \x \y) -- (chip \xx \yy);
				
				\pgfmathtruncatemacro{\xx}{\x+0}
				\pgfmathtruncatemacro{\yy}{\y-1}
				\draw (chip \x \y) -- (chip \xx \yy);
			}
		}
	\end{scope}
	
	% Hide existing link and redraw as dead link
	\draw [ultra thick, white] (chip 43) -- (chip 53);
	\draw [->, gray, thick, dotted]
	      (chip 43)
	   -- coordinate (dead link center)
	      (chip 53);
	\node [red] at (dead link center) {$\times$};
	
	% Draw emergency route 1
	\draw [ultra thick, white] (chip 43) -- (chip 54);
	\draw [->, thick]
	      (chip 43)
	   -- coordinate (first link center)
	      (chip 54);
	\node [font=\tiny] at (first link center) {\contour{white}{1}};
	
	% Draw emergency route 2
	\draw [ultra thick, white] (chip 54) -- (chip 53);
	\draw [->, thick]
	      (chip 54)
	   -- coordinate (second link center)
	      (chip 53);
	\node [font=\tiny] at (second link center) {\contour{white}{2}};
	
\end{tikzpicture}


				
				\caption[Emergency routing example.]{Emergency routing example. A
				faulty link is avoided by traversing two neighbouring links.}
				\label{fig:emergency-routing}
			\end{figure}
			
			An alternative to attempting to use links during PLL acquisition is to
			redirect traffic via another route in the system. One of the advantages of
			the hexagonal torus topology is a high level of redundancy between links.
			At the chip-to-chip level, SpiNNaker supports an `emergency routing'
			scheme where packets are automatically re-routed around faulty links as
			shown in figure \ref{fig:emergency-routing}. These emergency routes
			feature only a single hop of overhead and additionally require only very
			simple, fixed routing. Since the HSS board-to-board links within SpiNNaker
			are also arranged in a (coarser) hexagonal torus, this same principle can
			in be used here.
			
			\begin{figure}
				\center
				\begin{tikzpicture}[thick]
	\pgfmathsetlengthmacro{\routeroffset}{2em}
	\pgfmathsetlengthmacro{\routerheight}{2em}
	
	\def\boardoffset{3}
	
	\newcommand{\spinnlink}[4]{
		% Block
		\draw ([shift={(#3:\routeroffset)}]#1) coordinate (#4 p0) -- coordinate (#4 bottom)
		      ([shift={(#3:\routeroffset)}]#2) coordinate (#4 p1) -- coordinate (#4 right)
		      ([shift={(#3:\routeroffset+\routerheight)}]#2) coordinate (#4 p2) -- coordinate (#4 top)
		      ([shift={(#3:\routeroffset+\routerheight)}]#1) coordinate (#4 p3) -- coordinate (#4 left)
		      ([shift={(#3:\routeroffset)}]#1)
		   -- cycle
		      ;
		% Links to chips
		\foreach \link in {0.0,0.14285714,...,1.0}{
			\draw ($(#1)!\link!(#2)$) -- ++(#3:\routeroffset);
		}
		
		% Links to other boards
		\draw ([shift={(#3:\routeroffset+\routerheight)}]$(#1)!0.5!(#2)$)
		   -- ++([shift={(#3:-\routeroffset-\routerheight)}]#3:\boardoffset);
	}
	
	\begin{scope}[scale=0.3, help lines]
		\clip (6,6.5) rectangle ++(39.5,24);
		\foreach \tbyoff in {0,...,1}{
			\foreach \tbxoff in {0,...,2}{
				\foreach \xoff/\yoff/\colour in {%
					0                / 0                /red,%
					4+\boardoffset   / 8+\boardoffset*2 /green,%
					8+\boardoffset*2 / 4+\boardoffset   /blue%
				}{
					% Chips
					\foreach \x/\y/\num in {%
					                                          0/4/45,   1/4/46,   2/4/47,   3/4/48, %
					                                -1/3/44,  0/3/25,   1/3/26,   2/3/27,   3/3/28, %
					                      -2/2/43,  -1/2/24,  0/2/11,   1/2/12,   2/2/13,   3/2/29, %
					            -3/1/42,  -2/1/23,  -1/1/10,  0/1/3,    1/1/4,    2/1/14,   3/1/30, %
					  -4/0/41,  -3/0/22,  -2/0/9,   -1/0/2,   0/0/1,    1/0/5,    2/0/15,   3/0/31, %
					  -4/-1/40, -3/-1/21, -2/-1/8,  -1/-1/7,  0/-1/6,   1/-1/16,  2/-1/32,          %
					  -4/-2/39, -3/-2/20, -2/-2/19, -1/-2/18, 0/-2/17,  1/-2/33,                    %
					  -4/-3/38, -3/-3/37, -2/-3/36, -1/-3/35, 0/-3/34,                              %
					}{
						\begin{scope}[hexagonXYZ]
							\node [ inner sep=0
							      , draw
							      , hexagon
							      , minimum size=0.3cm
							      , inner sep=0
							      ]
							      at ( \x+\xoff+\tbxoff*12+\tbxoff*\boardoffset*3
							         , \y+\yoff+\tbyoff*12+\tbyoff*\boardoffset*3)
							      (x\tbxoff y\tbyoff z\colour chip\num)
							      {};
						\end{scope}
					}
					
					% Links
					\spinnlink{x\tbxoff y\tbyoff z\colour chip45.side north}{     x\tbxoff y\tbyoff z\colour chip48.side north east}{90}{x\tbxoff y\tbyoff z\colour fpganorth};
					\spinnlink{x\tbxoff y\tbyoff z\colour chip41.side north}{     x\tbxoff y\tbyoff z\colour chip45.side west}{150}{x\tbxoff y\tbyoff z\colour fpgawest};
					\spinnlink{x\tbxoff y\tbyoff z\colour chip38.side south west}{x\tbxoff y\tbyoff z\colour chip41.side west}{-150}{x\tbxoff y\tbyoff z\colour fpgasouth west};
					\spinnlink{x\tbxoff y\tbyoff z\colour chip34.side south west}{x\tbxoff y\tbyoff z\colour chip38.side south}{-90}{x\tbxoff y\tbyoff z\colour fpgasouth};
					\spinnlink{x\tbxoff y\tbyoff z\colour chip31.side east}{      x\tbxoff y\tbyoff z\colour chip34.side south}{-30}{x\tbxoff y\tbyoff z\colour fpgaeast};
					\spinnlink{x\tbxoff y\tbyoff z\colour chip48.side east}{      x\tbxoff y\tbyoff z\colour chip31.side north east}{30}{x\tbxoff y\tbyoff z\colour fpganorth east};
					
					% Connect links together
					%\draw (x\tbxoff y\tbyoff z\colour fpganorth left)      -- (x\tbxoff y\tbyoff z\colour fpgawest right);
					\draw (x\tbxoff y\tbyoff z\colour fpgawest left)       -- (x\tbxoff y\tbyoff z\colour fpgasouth west right);
					%\draw (x\tbxoff y\tbyoff z\colour fpgasouth west left) -- (x\tbxoff y\tbyoff z\colour fpgasouth right);
					\draw (x\tbxoff y\tbyoff z\colour fpgasouth left)      -- (x\tbxoff y\tbyoff z\colour fpgaeast right);
					%\draw (x\tbxoff y\tbyoff z\colour fpgaeast left)       -- (x\tbxoff y\tbyoff z\colour fpganorth east right);
					\draw (x\tbxoff y\tbyoff z\colour fpganorth east left) -- (x\tbxoff y\tbyoff z\colour fpganorth right);
				}
			}
		}
	\end{scope}
	
	% Dead link
	\draw [ultra thick, white]
	      (x1y0zgreenfpganorth east top)
	   -- (x2y1zredfpgasouth west top);
	\draw [->,dotted,gray]
	      (x1y0zgreenfpganorth east top) -- coordinate (mid dead link)
	      (x2y1zredfpgasouth west top);
	\node [red] at (mid dead link) {$\times$};
	
	%% Source chip
	%\node [draw,hexagon,inner sep=0,minimum size=0.3cm] at (x1y0zgreenchip30) {};
	%
	%% Source link
	%\draw [->] (x1y0zgreenchip30.side north east) -- ++(30:0.3*\routeroffset);
	
	% Source link outline
	\draw (x1y0zgreenfpganorth east p0)
	   -- (x1y0zgreenfpganorth east p1)
	   -- (x1y0zgreenfpganorth east p2)
	   -- (x1y0zgreenfpganorth east p3)
	   -- cycle;
	
	% Inter-FPGA link 1
	\draw [->] (x1y0zgreenfpganorth east left) -- (x1y0zgreenfpganorth right);
	
	% Source second link outline
	\draw (x1y0zgreenfpganorth p0)
	   -- (x1y0zgreenfpganorth p1)
	   -- (x1y0zgreenfpganorth p2)
	   -- (x1y0zgreenfpganorth p3)
	   -- cycle;
	
	% Board-to-board link 1
	\draw [->] (x1y0zgreenfpganorth top) -- (x1y1zbluefpgasouth top);
	
	% Intermediate first link outline
	\draw (x1y1zbluefpgasouth p0)
	   -- (x1y1zbluefpgasouth p1)
	   -- (x1y1zbluefpgasouth p2)
	   -- (x1y1zbluefpgasouth p3)
	   -- cycle;
	
	% Intermediate second link outline
	\draw (x1y1zbluefpgaeast p0)
	   -- (x1y1zbluefpgaeast p1)
	   -- (x1y1zbluefpgaeast p2)
	   -- (x1y1zbluefpgaeast p3)
	   -- cycle;
	
	% Inter-FPGA link 2
	\draw [->] (x1y1zbluefpgasouth left) -- (x1y1zbluefpgaeast right);
	
	% Board-to-board link 2
	\draw [->] (x1y1zbluefpgaeast top) -- (x2y1zredfpgawest top);
	
	% Dest first link outline
	\draw (x2y1zredfpgawest p0)
	   -- (x2y1zredfpgawest p1)
	   -- (x2y1zredfpgawest p2)
	   -- (x2y1zredfpgawest p3)
	   -- cycle;
	
	% Inter-FPGA link 3
	\draw [->] (x2y1zredfpgawest left) -- (x2y1zredfpgasouth west right);
	
	% Dest second link outline
	\draw (x2y1zredfpgasouth west p0)
	   -- (x2y1zredfpgasouth west p1)
	   -- (x2y1zredfpgasouth west p2)
	   -- (x2y1zredfpgasouth west p3)
	   -- cycle;
	
	%% Dest link
	%\draw [<-] (x2y1zredchip39.side south west) -- ++(30:-0.3*\routeroffset);
	%
	%% Dest chip
	%\node [draw,hexagon,inner sep=0,minimum size=0.3cm] at (x2y1zredchip39) {};
	
\end{tikzpicture}	

				
				\caption[Board-to-board traffic redirection.]{Board-to-board traffic
				redirection. An unavailable link (e.g. when faulty, PLL-locking or
				powered down) is avoided by traversing two neighbouring HSS links. Each
				group of hexagons represents a board of SpiNNaker chips (drawn as
				hexagons), each rectangular block is a board-to-board FPGA-based HSS
				block. Connected HSS blocks associated with the same board share an FPGA
				and may communicate cheaply. Lines between boards are HSS links.}
				\label{fig:board-to-board-redirection}
			\end{figure}
			
			Figure \ref{fig:board-to-board-redirection} illustrates how frames
			destined for an unavailable HSS link are instead transmitted via a
			neighbouring link.  Though this alternative route appears much longer,
			only two HSS links are crossed (an overhead of one HSS hop). Since each
			FPGA implements two HSS board-to-board links, frames destined for an
			unavailable link can be cheaply forwarded to the other link for
			transmission. The intermediate FPGA then dumbly forwards the frame to its
			other HSS link where the frame finally arrives at the intended FPGA. The
			arriving frame must then be forwarded within the FPGA to the unavailable
			link's block which may then process the frame as if it had arrived
			directly.
			
			As with emergency routing, the redirection technique only works when only
			one of the HSS links within a set of three is unavailable. As a result
			link speed changes must be carefully scheduled in order to ensure that
			only one link within the set is changing speed at a given moment.
		
		\subsection{`Santos-28' test chip}
			
			% A 28 nm test chip is being developed with TU Dresden to evaluate various
			% technologies' performance for inclusion in a second generation SpiNNaker
			% chip. There may be room in this space for my own involvement in this
			% work at short notice.
			
			In a collaboration between the University of Manchester and the Technical
			University of Dresden a test chip is being developed to pave the way
			towards the next generation of SpiNNaker system. The test chip will be
			manufactured using a modern 28 nm process and will notably make use of HSS
			links both internally and also between chips. The proposed HSS links have
			been developed by the researchers in Dresden and therefore work in this
			area may be possible. Due to the currently unpredictable state of this
			project, this work does not appear in the Gantt chart. Should the work
			take place, it will initially replace the other work planned in this
			section.
	
	\section{Thesis writing}
		
		A large block of time is reserved for the production of a thesis. This
		generous allocation ensures that adequate time is available for the
		collection of additional data which are found to be missing at the time of
		writing.
