\chapter{Research plan}
	\label{sec:research-plan}
	
	\begin{figure}[b!]
		\center
		\input{figures/plan-gantt}
		\caption[Gantt chart overview of research plan.]{Gantt chart overview of
		research plan. Boxes indicate expected duration of a task, thick lines
		indicate slack and red arrows show dependencies between tasks. Note the
		non-linear scale.}
		\label{fig:plan-gantt}
	\end{figure}
	
	The preliminary work so-far has focused on the analysis and exploration of
	extensions to SpiNNaker's interconnection network, in particular on the
	high-speed serial links which connect chips on different boards within the
	system. Based on the wire-length limitations imposed by HSS technology, a new
	wiring scheme was developed which achieves both the desired network topology
	while also ensuring wires remained short. Following this development, this
	scheme was extended to support small-world connectivity between boards,
	potentially enabling reduced latencies within the existing SpiNNaker network.
	In preparation for implementation within SpiNNaker's hardware, a simulator has
	been developed to model the potential benefits and guide the design. Finally,
	the foundations of hardware extensions to SpiNNaker's board-to-board
	interconnect have been laid. The proof-of-concept system built on these
	foundations has already produced measurable improvements to certain SpiNNaker
	applications.
	
	The proposed research plan is summarised in figure \ref{fig:plan-gantt}. The
	work can be divided into two distinct parts. The first investigates the use of
	small-world, or small-world-like networks within actual SpiNNaker systems. The
	second extends this to attempt to improve power efficiency within SpiNNaker by
	powering-off or slowing down links where lightly-loaded alternatives exist.
	
	This work will form the basis of a new generation of HSS-based interconnect
	with applications both for existing systems such as SpiNNaker and the next
	generation of neural simulators. In particular, this work hopes to produce
	novel HSS power-management techniques suited to realtime applications and
	which take advantage of the topologies allowed by HSS networks.
	
	To measure success, performance will be benchmarked using realistic network
	traffic from both neural simulations taken from the field and synthetic loads.
	Since directly instrumenting many neural models can introduce unacceptable
	overheads, this task will exploit the properties of the high-level NEF neural
	modelling framework to inexpensively produce artificial network traffic which
	is identical to neural simulations. With the processing and storage resources
	this frees up a collection of suitable metrics can be gathered directly.
	
	This chapter will outline each of the key research aims and the work proposed
	in approximately chronological order.
	
	\section{NEF network experiments}
		
		As described earlier, the NEF provide an actively used neural modelling
		architecture capable of generating large scale neural models. Unlike many
		other frameworks the NEF is already being used to construct large models
		from simple neurons, the design target of SpiNNaker. The SPAUN model, for
		example, contains 2.5 million simple neurons \cite{eliasmith12}. As well as
		being large, NEF models generally feature easily verified high-level
		behaviour making it easy to identify faults. Another distinctive feature of
		the NEF is that, while biologically plausible, the pattern of communication
		between populations of neurons (in terms of the rate and pattern of spikes)
		can be reproduced inexpensively. It is this property which makes the NEF
		especially attractive for network benchmarking since it frees up processing
		and storage resources for the direct collection of statistics relating to
		network behaviour.
		
		The above factors all support the NEF as a strong choice for benchmarking
		the interconnection networks of neural modelling systems. In this section,
		the methods by which network communication patterns can be recreated are
		outlined. This is followed by a discussion of the measures and networks
		which will be used.
		
		\subsection{Recreating NEF spike traffic}
			
			In the NEF, populations of neurons represent continuous (analogue) vector
			values encoded by a linear combination of the neurons' firing rates. When
			neural populations are generated uniformly -- the case for most existing
			models -- the average rate of firing within a population remains
			approximately constant, regardless of the value encoded. Further, since
			simple LIF neurons are frequently used, the firing patterns exhibited are
			very regular. The exception to this rule is when an `inhibitory' signal is
			applied to a population which causes the neurons to cease firing.
			
			As described in \S\ref{sec:spinn-router}, spikes within a neural simulator
			are routed at the granularity of whole populations. This means that
			traffic in the network is proportional to the aggregate behaviour of
			populations of neurons since every neuron impacts the network in the same
			manner. As a result, the network traffic due to a population of neurons in
			an NEF network can be accurately approximated by a bank of Poisson sources
			whose average firing rate matches that of the population. To model
			inhibitory connections, recorded traces of the inhibitory signals from
			simulations may be used to control the generation of random packets.
		
		\subsection{Recreating NEF vector traffic}
			
			In practice, many simulators implemented for the NEF don't transmit spikes
			through their networks. Instead, the vector which the spikes encode is
			transmitted. For NEF models this approach is provably equivalent to
			transmitting spikes but can save both network, memory and processing
			resources. Though the memory and processing savings are beyond the scope
			of this report, the network saving is described below in the case of the
			SpiNNaker implementation of the NEF.
			
			Network savings occur since a population with many hundreds of neurons
			will typically represent a vector fewer than ten values. As an example,
			consider a population of 1,000 neurons used to represent a two-valued
			vector. If each neuron fires at a biologically representative rate of 10
			Hz, the population will produce around ten spikes each millisecond time
			step resulting in the transmission of a total of ten 40-bit SpiNNaker
			packets (400 bits). If the vector is transmitted instead, it must be
			transmitted every time step and requires one 72 bit SpiNNaker packet for
			each vector element (since a single element will fit in the payload of a
			SpiNNaker packet). This results in a total of 144 bits being transmitted
			per time step.
			
			Clearly, this mode of operation results in trivial to reproduce network
			usage since the number of transmissions per time step is exactly uniform
			and constant, irrespective of the values represented or inhibitory
			signals. The timing of transmissions within a time step is determined by
			the simulator used.
			
			This mode of operation doesn't match the resources available in hardware
			simulation platforms such as SpiNNaker since the transmission of of
			continuous bursts of data (i.e. vectors) is not supported. Though it is
			possible to work around this limitation as described above, it is clear
			that the omission of such burst transmission modes is a shortcoming of
			such architectures which will be considered in the work described in this
			section.
		
		\subsection{Benchmark networks and metrics}
			
			A number of benchmark networks utilising the NEF will be collected from
			the literature. Though a specific set of networks remains to be assembled,
			candidates include SPAUN \cite{eliasmith12}, the Basal Ganglia
			\cite{stewart12} and models produced during the 2014 Nengo Summer School.
			
			Additionally, various synthetic benchmark networks will be produced aimed
			to stress certain special-cases of network connectivity. Candidates
			include all-to-all connected networks, networks with highly regular
			structure (e.g. trees, meshes) and random connectivity. These synthetic
			networks may be scaled to arbitrarily large systems to evaluate networks
			larger than currently studied in the literature but which may be handled
			by existing hardware.
			
			Though other metrics may developed as necessary, the following
			will be recorded during benchmark runs.
			
			\begin{description}
				
				\item[Packet injection rate] gives a basic indicator of the loading of
				the network. Recordings will attempt to measure a histogram of injection
				rates over the course of a simulation time step to determine how evenly
				load is injected into the network.
				
				\item[Packet dropping rate] indicates instances of congestion and
				deadlock within the network. By recording the location of dropped
				packets and their intended route, hot-spots in the network can be
				detected.
				
				\item[Packet routing rate] indicates the amount of traffic passing
				through a given chip. This in turn gives an indication of load in each
				part of the network.
				
				\item[Packet latency] is a critical measure for realtime neural
				simulations. If latency grows sufficiently large, models will behave
				incorrectly.
				
				\item[Packet arrival jitter] (variation in latency) can help inform the
				safety margins required when transmitting a packet to meet a realtime
				deadline.
				
			\end{description}
			
			To ease the conversion of existing neural models into network traffic
			experiments, the ability to globally change the underlying neuron model
			used by a network within NEF networks on SpiNNaker will be used. A new
			`neuron' model will be developed which implements the traffic generation
			schemes proposed in this section along with the reporting of the above
			metrics. This technique will mean that models can be used in network
			experiments with a single changed line of code.
		
		\subsection{SpiNNaker NEF improvements}
			
			In addition to providing a suite of benchmarks to validate later work,
			this suite of benchmarks will initially be used to contribute to the
			implementation of the NEF on SpiNNaker. This work will be carried out in
			collaboration with the researchers responsible for maintaining the
			SpiNNaker NEF implementation.
			
			This work will make two key contributions. The first will be to explicitly
			define what classes of network can be successfully mapped into SpiNNaker.
			This would be the first such work to explore this problem and it is could
			inform the design of future neural simulations systems.
			
			The second contribution will be to determine the impact that the timing of
			traffic injection into SpiNNaker's network on network performance. Within
			each simulation time step, packets must be inserted into the network with
			enough time to allow them to arrive at their destination. At present,
			packets are transmitted immediately, as soon as computation has completed
			potentially resulting in bursts of network activity increasing the chances
			of deadlock in SpiNNaker-like networks \cite{dally04}. This work will
			attempt to determine the effects of distributing packet transmissions in
			an alternative manner, for example distributing uniformly or randomly
			during a simulation time step.
			
			These contributions are planned for Journal publication around the end of
			the academic year.
	
	\section{Small-world networks for SpiNNaker}
		
		% Develop routing scheme for SpiNNaker packets suitable for peripheral, host
		% and preliminary small-world communications via the SpiNNaker links. This
		% scheme will continue to make use of the current HSS protocol blocks.
		
		To exploit the benefits of small-world network topologies, these will be
		implemented within SpiNNaker's board-to-board interconnection network. Work
		will begin by extending the Tickysim network simulator with small-world
		connectivity and a model of SpiNNaker's board-to-board links. Guided by
		these simulation results, this will be followed by an implementation in
		SpiNNaker.  This implementation will then be evaluated using the NEF
		benchmarks described previously.
		
		There are three main challenges associated with an implementation of
		small-world networks in SpiNNaker which are outlined in this section. The
		first is the construction of a routing scheme which can be implemented
		practically within the available FPGA logic. The second is deciding the
		method by which random links are added and what effect this has on physical
		assembly difficulties and performance. The final challenge, which will be
		undertaken only if time allows, is the possibility of routing packets
		between FPGAs on the same board. This would allow, for example, packets to
		`hop over' an entire board instead of traversing the every chip along the
		path through the board.
		
		\subsection{Routing}
			
			One of the principal challenges of implementing new network topologies
			within the FPGAs on a SpiNNaker board is that all routing must take place
			within the FPGA itself. Within a SpiNNaker chip, routing is handled by
			dedicated routing hardware featuring a generous ternary CAM which may
			prove expensive within FPGA logic \cite{locke11}. Additionally, each FPGA
			is responsible for sixteen individual streams of traffic, all of which
			must be routed without a reduction in throughput.
			
			In preliminary work, a very simple, single-entry key and mask routing
			scheme proved adequate for simple host connectivity.  This scheme consists
			of a mask and comparison performed on each packet to determine its route
			which can be implemented very cheaply within the FPGA logic.  This
			approach is adequate here because, generally, there will be only a few I/O
			devices in a system compared with the number of neurons and so it is
			feasible to allocate a bit in routing keys to indicate I/O packets.  For
			small-world networks, however, allocating a bit in the key space to
			indicate when each small-world link should be used would require more bits
			than are available in routing keys due to the greater number of
			small-world links.
			
			One possible solution to this problem is to exploit the relative speed
			differences between the FPGA logic and the links to SpiNNaker chips. A
			SpiNNaker chip link can nominally transmit a (short) packet in around 167
			$\mu$s while FPGA logic clocked at 75 MHz cycles in around 13 ns. This
			means that a router with a throughput of one packet per cycle could handle
			up to twelve links without becoming a bottleneck. This means a that only
			two routers would be required to handle all sixteen of each FPGA's links.
			A feasibility study will be required to establish the potential capability
			of such routers given the resources available on the FPGAs on each board.
			
			An alternative approach is to allow the FPGAs to modify the routing keys
			of packets which pass through them. If it is found that most packets will
			only use a single small-world link, a single bit could be defined within
			the key space to indicate that a packet should use the first small-world
			link it encounters. This bit would be cleared as it crosses the small
			world link after which the packet would be routed normally. This method
			would be able to exploit the existing key and mask routing scheme already
			implemented at the expense of greater complexity in the placement and
			routing system.
			
			As well as the technological factors described, the choice may be guided
			by time constraints since a basic implementation of the second approach is
			likely to be significantly less time consuming than a full FPGA based
			router implementation.
		
		\subsection{Random connectivity}
			
			The method by which random connectivity is introduced into small-world
			networks can have a significant impact on both the amount of manual labour
			required to install it and the performance gains achieved. One approach is
			to generate random connections and produce a list of wiring instructions
			for a technician to assemble. An alternative is to allow the technician to
			add the connections randomly at his or her own discretion.
			
			Computer generated random connections have the advantage that they can be
			both unbiased and may be post processed to prune non-beneficial links
			which, for example, happen to connect two already neighbouring or
			nearly-neighbouring boards. The main downside of this approach is that the
			effort required to assemble a large number of connections may be great,
			especially for large machines where there may be as many as 3,600
			potential connection points.
			
			The alternative approach, while straightforward for a technician's wiring
			will suffer from the poor quality of human-generated randomness
			\cite{figurska08}. Additionally, humans are unlikely to be able to spot
			and avoid `obviously poor' choices due to the complexity of the wiring in
			large systems.
			
			Some work is required to establish the efficacy of either approach. Simple
			experiments building on the simple graph based models used in preliminary
			work will be carried out to attempt to judge the importance of the
			randomness of decision making.
		
		\subsection{Board-hopping}
			
			% TODO: Diagram?
			
			The FPGAs on SpiNNaker boards are interconnected allowing them to
			communicate locally via a HSS ring network. If time allows, this could be
			used to forward packets directly which would otherwise pass through many
			chips the board before emerging on the other side. This mechanism will
			reduce contention for link resources on intervening chips while
			potentially reducing the latency for the packet forwarded. This mechanism
			is a variant of an express cube \cite{dally91} which exploits existing
			unused hierarchy in the network.
		
		\subsection{Evaluation}
			
			The suite of NEF benchmark networks described in the previous section will
			be used to evaluate the performance of the system. In particular, it is
			expected that the load throughout the network, measured in terms of the
			rate at which packets pass through every node will be reduced in the
			presence of small-world links. Additionally, latency is expected to be
			reduced for long connections within the network.
			
			It is consequently hypothesised that larger and more heavily connected
			networks will therefore be supported. This hypothesis will be tested by
			varying the size of synthetic networks to determine if larger networks are
			enabled by small-world connectivity.
	
	
	\section{USB 3.0 interface}
		
		As part of the collaboration started at the Capo Caccia workshop to develop
		a high speed interface for communications with SpiNNaker, a USB 3.0 system
		will be produced. This work is awaiting the completion of a circuit board
		design on which the interface will be implemented. No firm date has been
		given for arrival of a prototype board and so the dates indicated in figure
		\ref{fig:plan-gantt} are approximate. This work will consist of an extension
		of the UART proof-of-concept FPGA design which interfaces with a high-speed
		Cypress FX-3 USB 3.0 interface chip in place of the low-speed FTDI
		UART-over-USB interface chip.
		
		Given the timely availability of this device, it will allow superior
		monitoring of the SpiNNaker network since packets travelling over a link can
		be `snooped' via the USB link without impacting the system's performance.
		Additionally, it will allow software models of high-bandwidth I/O devices
		and host communications to be implemented.
	
	
	\section{High-speed serial power management}
		
		% Work to try and make some power savings by powering down idle links or
		% reducing speed (to reduce link transitions), taking into account latency
		% concerns for SpiNNaker.
		
		Power consumption is an important consideration for large SpiNNaker
		machines. At present, SpiNNaker's high-speed serial links operate at full
		speed at all times, even when no traffic is present, potentially wasting a
		significant amount of energy. While numerous techniques for power managing
		HSS technologies exist, these tend to be optimised toward more conventional
		supercomputer loads.
		
		In general, power savings are achieved either by powering down idle or
		underutilised links or by running them at low speeds. These techniques
		introduce problems for latency-sensitive, lightly-loaded interconnection
		networks such as SpiNNaker. Powering up or changing link speeds introduces
		an inherent delay since it requires the sender and receiver's CRC to re-lock
		on to the new frequency which can take on the order of a millisecond
		\cite{xilinx14}. Though this latency may be acceptable for some systems, it
		would violate the SpiNNaker realtime requirement that packets are delivered
		within 1 ms.
		
		This section describes two complementary proposals for exploiting these
		existing low-power techniques while also maintaining realtime performance.
		The section concludes with a possible collaboration with the Technical
		University of Dresden where a link with better power-up or speed-change
		performance may be developed.
		
		\subsection{Link oversampling}
			
			% During PLL locking, could still have good performance
			
			During clock recovery, the recovered clock signal is initially of low
			quality and gradually approaches the original clock signal. It is
			hypothesised is that during this acquisition period, data may still be
			sent, but at low speeds. This could be achieved, for example, by
			duplicating every bit sent across the link (oversampling) and thus
			reducing sensitivity to the poor quality clock signal.  Further error
			detection information may also be included during this time to handle the
			increased error rate. This technique could reduce the latency caused by
			link speed changes by allowing useful data to pass across links earlier.
			
			A large part of this work will consist of characterising the quality of
			the link during clock recovery and thus determining the degree of
			oversampling required.
		
		\subsection{Traffic redirection}
			
			% Send traffic around other edges of a triangle
			
			\begin{figure}
				\center
				\input{figures/emergency-routing}
				
				\caption[Emergency routing example.]{Emergency routing example. A
				faulty link is avoided by traversing two neighbouring links.}
				\label{fig:emergency-routing}
			\end{figure}
			
			An alternative to attempting to use links during clock recovery is to
			redirect traffic via an alternative route. One of the advantages of the
			hexagonal torus topology is a high level of redundancy in routing
			possibilities.  At the chip-to-chip level, SpiNNaker exploits this in its
			`emergency routing' scheme where packets are automatically re-routed
			around faulty links as shown in figure \ref{fig:emergency-routing}. These
			emergency routes feature only a single hop of overhead and require only
			simple, fixed routing rules. Since the HSS board-to-board links within
			SpiNNaker are also arranged in a (coarser) hexagonal torus, the same
			principle can be used here.
			
			\begin{figure}
				\center
				\begin{tikzpicture}[thick]
	\pgfmathsetlengthmacro{\routeroffset}{2em}
	\pgfmathsetlengthmacro{\routerheight}{2em}
	\pgfmathsetlengthmacro{\fpgagroupoffset}{1em}
	
	\def\boardoffset{3}
	
	\newcommand{\spinnlink}[4]{
		% Block
		\draw ([shift={(#3:\routeroffset)}]#1) coordinate (#4 p0) -- coordinate (#4 bottom)
		      ([shift={(#3:\routeroffset)}]#2) coordinate (#4 p1) -- coordinate (#4 right)
		      ([shift={(#3:\routeroffset+\routerheight)}]#2) coordinate (#4 p2) -- coordinate (#4 top)
		      ([shift={(#3:\routeroffset+\routerheight)}]#1) coordinate (#4 p3) -- coordinate (#4 left)
		      ([shift={(#3:\routeroffset)}]#1)
		   -- cycle
		      ;
		% Links to chips
		\foreach \link in {0.0,0.14285714,...,1.0}{
			\draw ($(#1)!\link!(#2)$) -- ++(#3:\routeroffset);
		}
		
		% Links to other boards
		\draw ([shift={(#3:\routeroffset+\routerheight)}]$(#1)!0.5!(#2)$)
		   -- ++([shift={(#3:-\routeroffset-\routerheight)}]#3:\boardoffset);
	}
	
	\begin{scope}[scale=0.3, help lines]
		\clip (6,6.5) rectangle ++(39.5,24);
		\foreach \tbyoff in {0,...,1}{
			\foreach \tbxoff in {0,...,2}{
				\foreach \xoff/\yoff/\colour in {%
					0                / 0                /red,%
					4+\boardoffset   / 8+\boardoffset*2 /green,%
					8+\boardoffset*2 / 4+\boardoffset   /blue%
				}{
					% Chips
					\foreach \x/\y/\num in {%
					                                          0/4/45,   1/4/46,   2/4/47,   3/4/48, %
					                                -1/3/44,  0/3/25,   1/3/26,   2/3/27,   3/3/28, %
					                      -2/2/43,  -1/2/24,  0/2/11,   1/2/12,   2/2/13,   3/2/29, %
					            -3/1/42,  -2/1/23,  -1/1/10,  0/1/3,    1/1/4,    2/1/14,   3/1/30, %
					  -4/0/41,  -3/0/22,  -2/0/9,   -1/0/2,   0/0/1,    1/0/5,    2/0/15,   3/0/31, %
					  -4/-1/40, -3/-1/21, -2/-1/8,  -1/-1/7,  0/-1/6,   1/-1/16,  2/-1/32,          %
					  -4/-2/39, -3/-2/20, -2/-2/19, -1/-2/18, 0/-2/17,  1/-2/33,                    %
					  -4/-3/38, -3/-3/37, -2/-3/36, -1/-3/35, 0/-3/34,                              %
					}{
						\begin{scope}[hexagonXYZ]
							\node [ inner sep=0
							      , draw
							      , hexagon
							      , minimum size=0.3cm
							      , inner sep=0
							      ]
							      at ( \x+\xoff+\tbxoff*12+\tbxoff*\boardoffset*3
							         , \y+\yoff+\tbyoff*12+\tbyoff*\boardoffset*3)
							      (x\tbxoff y\tbyoff z\colour chip\num)
							      {};
						\end{scope}
					}
					
					% Links
					\spinnlink{x\tbxoff y\tbyoff z\colour chip45.side north}{     x\tbxoff y\tbyoff z\colour chip48.side north east}{90}{x\tbxoff y\tbyoff z\colour fpganorth};
					\spinnlink{x\tbxoff y\tbyoff z\colour chip41.side north}{     x\tbxoff y\tbyoff z\colour chip45.side west}{150}{x\tbxoff y\tbyoff z\colour fpgawest};
					\spinnlink{x\tbxoff y\tbyoff z\colour chip38.side south west}{x\tbxoff y\tbyoff z\colour chip41.side west}{-150}{x\tbxoff y\tbyoff z\colour fpgasouth west};
					\spinnlink{x\tbxoff y\tbyoff z\colour chip34.side south west}{x\tbxoff y\tbyoff z\colour chip38.side south}{-90}{x\tbxoff y\tbyoff z\colour fpgasouth};
					\spinnlink{x\tbxoff y\tbyoff z\colour chip31.side east}{      x\tbxoff y\tbyoff z\colour chip34.side south}{-30}{x\tbxoff y\tbyoff z\colour fpgaeast};
					\spinnlink{x\tbxoff y\tbyoff z\colour chip48.side east}{      x\tbxoff y\tbyoff z\colour chip31.side north east}{30}{x\tbxoff y\tbyoff z\colour fpganorth east};
					
					% Connect links together
					%\draw (x\tbxoff y\tbyoff z\colour fpganorth left)      -- (x\tbxoff y\tbyoff z\colour fpgawest right);
					\draw (x\tbxoff y\tbyoff z\colour fpgawest left)       -- (x\tbxoff y\tbyoff z\colour fpgasouth west right);
					%\draw (x\tbxoff y\tbyoff z\colour fpgasouth west left) -- (x\tbxoff y\tbyoff z\colour fpgasouth right);
					\draw (x\tbxoff y\tbyoff z\colour fpgasouth left)      -- (x\tbxoff y\tbyoff z\colour fpgaeast right);
					%\draw (x\tbxoff y\tbyoff z\colour fpgaeast left)       -- (x\tbxoff y\tbyoff z\colour fpganorth east right);
					\draw (x\tbxoff y\tbyoff z\colour fpganorth east left) -- (x\tbxoff y\tbyoff z\colour fpganorth right);
					
					% Draw boxes around links shared by an FPGA
					\foreach \lfpga/\rfpga in {south west /west ,east /south ,north /north east }{
						\draw [dotted,thick,rounded corners=0.25*\fpgagroupoffset]
						      % Right FPGA:
						        % Bottom left
						        ($(x\tbxoff y\tbyoff z\colour fpga\rfpga p0)!-\fpgagroupoffset!(x\tbxoff y\tbyoff z\colour fpga\rfpga p3)$)
						        % Bottom right
						   --   ($(x\tbxoff y\tbyoff z\colour fpga\rfpga p1)!-\fpgagroupoffset!(x\tbxoff y\tbyoff z\colour fpga\rfpga p2)$)
						        % Right bottom
						   --   ($(x\tbxoff y\tbyoff z\colour fpga\rfpga p1)!-\fpgagroupoffset!(x\tbxoff y\tbyoff z\colour fpga\rfpga p0)$)
						        % Right top
						   --   ($(x\tbxoff y\tbyoff z\colour fpga\rfpga p2)!-\fpgagroupoffset!(x\tbxoff y\tbyoff z\colour fpga\rfpga p3)$)
						        % Top right
						   --   ($(x\tbxoff y\tbyoff z\colour fpga\rfpga p2)!-\fpgagroupoffset!(x\tbxoff y\tbyoff z\colour fpga\rfpga p1)$)
						        % Top left
						   --   ($(x\tbxoff y\tbyoff z\colour fpga\rfpga p3)!-\fpgagroupoffset!(x\tbxoff y\tbyoff z\colour fpga\rfpga p0)$)
						      % Left FPGA:
						        % Top right
						   --   ($(x\tbxoff y\tbyoff z\colour fpga\lfpga p2)!-\fpgagroupoffset!(x\tbxoff y\tbyoff z\colour fpga\lfpga p1)$)
						        % Top left
						   --   ($(x\tbxoff y\tbyoff z\colour fpga\lfpga p3)!-\fpgagroupoffset!(x\tbxoff y\tbyoff z\colour fpga\lfpga p0)$)
						        % Left top
						   --   ($(x\tbxoff y\tbyoff z\colour fpga\lfpga p3)!-\fpgagroupoffset!(x\tbxoff y\tbyoff z\colour fpga\lfpga p2)$)
						        % Left bottom
						   --   ($(x\tbxoff y\tbyoff z\colour fpga\lfpga p0)!-\fpgagroupoffset!(x\tbxoff y\tbyoff z\colour fpga\lfpga p1)$)
						        % Bottom left
						   --   ($(x\tbxoff y\tbyoff z\colour fpga\lfpga p0)!-\fpgagroupoffset!(x\tbxoff y\tbyoff z\colour fpga\lfpga p3)$)
						        % Bottom right
						   --   ($(x\tbxoff y\tbyoff z\colour fpga\lfpga p1)!-\fpgagroupoffset!(x\tbxoff y\tbyoff z\colour fpga\lfpga p2)$)
						   -- cycle
						      ;
					}
				}
			}
		}
	\end{scope}
	
	% Dead link
	\draw [ultra thick, white]
	      (x1y0zgreenfpganorth east top)
	   -- (x2y1zredfpgasouth west top);
	\draw [->,dotted,gray]
	      (x1y0zgreenfpganorth east top) -- coordinate (mid dead link)
	      (x2y1zredfpgasouth west top);
	\node [red] at (mid dead link) {$\times$};
	
	%% Source chip
	%\node [draw,hexagon,inner sep=0,minimum size=0.3cm] at (x1y0zgreenchip30) {};
	%
	%% Source link
	%\draw [->] (x1y0zgreenchip30.side north east) -- ++(30:0.3*\routeroffset);
	
	% Source link outline
	\draw (x1y0zgreenfpganorth east p0)
	   -- (x1y0zgreenfpganorth east p1)
	   -- (x1y0zgreenfpganorth east p2)
	   -- (x1y0zgreenfpganorth east p3)
	   -- cycle;
	
	% Inter-FPGA link 1
	\draw [->] (x1y0zgreenfpganorth east left) -- (x1y0zgreenfpganorth right);
	
	% Source second link outline
	\draw (x1y0zgreenfpganorth p0)
	   -- (x1y0zgreenfpganorth p1)
	   -- (x1y0zgreenfpganorth p2)
	   -- (x1y0zgreenfpganorth p3)
	   -- cycle;
	
	% Board-to-board link 1
	\draw [->] (x1y0zgreenfpganorth top) -- (x1y1zbluefpgasouth top);
	
	% Intermediate first link outline
	\draw (x1y1zbluefpgasouth p0)
	   -- (x1y1zbluefpgasouth p1)
	   -- (x1y1zbluefpgasouth p2)
	   -- (x1y1zbluefpgasouth p3)
	   -- cycle;
	
	% Intermediate second link outline
	\draw (x1y1zbluefpgaeast p0)
	   -- (x1y1zbluefpgaeast p1)
	   -- (x1y1zbluefpgaeast p2)
	   -- (x1y1zbluefpgaeast p3)
	   -- cycle;
	
	% Inter-FPGA link 2
	\draw [->] (x1y1zbluefpgasouth left) -- (x1y1zbluefpgaeast right);
	
	% Board-to-board link 2
	\draw [->] (x1y1zbluefpgaeast top) -- (x2y1zredfpgawest top);
	
	% Dest first link outline
	\draw (x2y1zredfpgawest p0)
	   -- (x2y1zredfpgawest p1)
	   -- (x2y1zredfpgawest p2)
	   -- (x2y1zredfpgawest p3)
	   -- cycle;
	
	% Inter-FPGA link 3
	\draw [->] (x2y1zredfpgawest left) -- (x2y1zredfpgasouth west right);
	
	% Dest second link outline
	\draw (x2y1zredfpgasouth west p0)
	   -- (x2y1zredfpgasouth west p1)
	   -- (x2y1zredfpgasouth west p2)
	   -- (x2y1zredfpgasouth west p3)
	   -- cycle;
	
	%% Dest link
	%\draw [<-] (x2y1zredchip39.side south west) -- ++(30:-0.3*\routeroffset);
	%
	%% Dest chip
	%\node [draw,hexagon,inner sep=0,minimum size=0.3cm] at (x2y1zredchip39) {};
	
\end{tikzpicture}	

				
				\caption[Board-to-board traffic redirection.]{Board-to-board traffic
				redirection. An unavailable link is avoided (e.g. when faulty, during
				initial clock recovery or powered down) by traversing two neighbouring
				HSS links. Each group of small hexagons (SpiNNaker chips) represents a
				board. Rectangular blocks represent board-to-board HSS blocks. Dotted
				lines indicate HSS blocks which share an FPGA and thus may communicate
				cheaply. Lines between boards are HSS links.}
				\label{fig:board-to-board-redirection}
			\end{figure}
			
			Figure \ref{fig:board-to-board-redirection} illustrates how frames
			destined for an unavailable HSS link are instead transmitted via a
			neighbouring link.  Though this alternative route appears much longer,
			only two HSS links are crossed (an overhead of one HSS hop). Since each
			FPGA implements two HSS board-to-board links, frames destined for an
			unavailable link can be forwarded cheaply within the FPGA to the other
			link for transmission. The intermediate FPGA then blindly forwards the
			frame over its other HSS link where the frame finally arrives at the
			intended FPGA. The arriving frame is then processed as if it had arrived
			directly.
			
			As with emergency routing, the redirection technique only works when at
			most one of the HSS links within a set of three is unavailable. As a
			result, link speed changes must be carefully scheduled to ensure that only
			one link within the set changes speed at a given moment.
		
		\subsection{`Santos-28' test chip}
			
			% A 28 nm test chip is being developed with TU Dresden to evaluate various
			% technologies' performance for inclusion in a second generation SpiNNaker
			% chip. There may be room in this space for my own involvement in this
			% work at short notice.
			
			In a collaboration between the University of Manchester and the Technical
			University of Dresden a test chip is being developed to pave the way
			towards the next generation of SpiNNaker system. The test chip will be
			manufactured using a modern 28 nm process and will notably make use of HSS
			links both internally and between chips. The proposed HSS links have been
			developed by the researchers in Dresden and therefore are potentially open
			to development. Due to the currently unpredictable state of this project,
			this work does not appear in the Gantt chart. Depending on the scope of
			this work, it may initially replace the other work planned in this
			section.
	
	\section{Thesis writing}
		
		A large block of time is reserved for the production of a thesis. This
		allocation ensures that adequate time is made available for the collection
		of additional data which are found to be missing at the time of writing.
