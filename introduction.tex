\chapter{Introduction}
	
	Modern computer systems are some of the most complex devices ever constructed.
	Current computer technologies have enabled everything from global,
	near-instantaneous communications via the internet to faster and more
	effective cancer treatments \cite{nassif}. Despite this, the brain still
	outperforms conventional digital computer in its ability to learn, tolerate
	faults and operate using little power.
	
	Considerable effort is being made by researchers across many disciplines to
	understand how the brain works. The small-scale, chemical operation of
	individual neurons in the brain is now relatively well understood but their
	complex interactions and emergent behaviour remain murky \cite{dayan03}. Many
	attempts to understand the brain's higher-level behaviour involve modelling
	huge networks of neurons and observing their collective behaviour.  Such
	models are challenging to simulate in traditional computer architectures since
	the vast parallelism and interconnection exhibited by neurons in the brain
	sharply contrasts with the highly serial, comparatively isolated computing
	resources available today.
	
	This report examines the challenges encountered when modelling the brain,
	focusing primarily on those faced by computer interconnection networks.
	Earlier parts of the report examine the underlying principles and impact of
	current modelling techniques. This is followed by background material
	describing the attempts being made to improve simulation performance and the
	importance of interconnect. The latter parts of the report outline the work
	being done to advance the state-of-the-art of neural simulator interconnect
	followed by plans for the continuation of this work.
	
	\section{Why model the brain?}
	
		Cutting-edge neural models can be broadly divided into two categories. The
		first is based on the behaviour of large networks of simplified models of
		neurons.  The second focusses on small numbers of extremely detailed neuron
		models.  Models of the former type, such as Spaun \cite{eliasmith12}, are
		able to demonstrate a remarkable range of cognitive abilities such as
		memory, problem solving and pattern recognition. Unlike non-neural models
		such as ACT-R \cite{anderson93}, Spaun exhibits similar responses to humans
		when faced with simulated ageing processes and illness. This level of
		realism could soon enable experiments which would not be possible on real
		brains, such as testing hypotheses of the causes of certain illnesses and
		possible treatments.
		
		% It also shows the same gradual reduction in overall function as individual
		% neurons die off, in sharp contrast with modern computers which typically
		% fail immediately with the loss of a single transistor. 
		
		In contrast the latter type of model has led to a much deeper understanding
		of biochemical processes in the brain. Though these models tend to feature
		great biological accuracy, they typically offer only limited insight into
		the larger-scale function of the brain.
		
		To computer scientists, the first type of model presents two particularly
		enticing properties, aside from the enhancement of biological understanding.
		The first is neural systems' exhibition of fault-tolerant, efficient
		high-level processing of vision and language from which much could be learnt
		in the field of computer design. The second property, and the focus of this
		work, is that large-scale models present a huge challenge to modern
		computing systems due to their sheer scale and communication requirements.
		Models such as Spaun, do not easily fit current supercomputer architectures
		and can take many hours of compute time to simulate one second of neural
		activity on a high-end workstation computer. This limited performance is
		stifling research in this field, highlighting the importance of new, faster
		approaches to neural simulation.
	
	\section{Conventional supercomputer approaches}
	
		Supercomputers such as Tianhe-2, currently the worlds
		fastest \cite{meuer13n}, are designed to provide immense computational power
		with quadrillions ($10^{15}$) of numerical calculations being performed per
		second. Despite such impressive feats of computation, these machines often
		feature interconnection networks tying internal processing elements
		together, with comparatively low performance. This balance of great
		computation and modest communication contrasts with the brain whose
		individual neurons offer relatively little computational power but are
		highly interconnected.
		
		This mismatch is evident in the simulation performance figures posted by
		leading large scale experiments. For example IBM, as part of DARPA's SyNAPSE
		programme, has constructed a model featuring $53 \times 10^{10}$ neurons
		connected via $1.37 \times 10^{14}$ synapses on a Blue Gene/Q supercomputer
		\cite{ibm13}. Though a formidable achievement, the model runs $1,542\times$
		slower than biological realtime and consumes $394,500\times$ more power
		than a similarly sized brain \cite{drubach00}. In the short term, this
		prevents the use of such models in robotics experiments which require
		realtime performance. In the longer term, developmental models would find
		themselves requiring multi-millennia runtimes just to reach the level of a
		toddler rendering such work impractical.
		
	\section{The importance of interconnect}
		
		Given the unsuitability of traditional supercomputer architectures, a number
		of specialised architectures have been developed. These architectures
		generally feature novel computational resources and, importantly, generous
		interconnection networks designed for neural simulation loads.
		
		The SpiNNaker project \cite{furber06} is one such architecture based on a
		custom interconnection network which is being scaled to over one million
		small, low-power computer processors. The purpose-built interconnection
		network is designed to handle neural simulations with up to one billion
		neurons and a total of one trillion neural connections (synapses) running in
		biological realtime.
		
		As the SpiNNaker machine is scaled up from hundreds of processors to one
		million, a new type of interconnect technology (`high-speed serial' (HSS))
		has been introduced to keep the construction of the machine practical. This
		technology has different qualities from SpiNNaker's original network and
		this research project aims to better understand and exploit this new
		resource.
		
		\section{Report structure}
			
			The report begins in chapter \ref{sec:background} by introducing the
			nature of neural modelling tools. This is followed by a discussion of the
			challenges faced by conventional supercomputer architectures in
			simulating these models. The chapter concludes by describing the SpiNNaker
			neural simulator, and the interconnection network technology it uses,
			which serves as a platform for this work.
			
			In chapter \ref{sec:preliminary-work} preliminary work is described to
			model, explore and propose extensions to SpiNNaker's interconnect. This
			work examines the physical wiring required by large simulation systems and
			proposes a novel extension to this wiring scheme turning the system into a
			`small-world' network with a number of attractive properties. Finally,
			foundational work on SpiNNaker's interconnect is described to allow both
			the implementation of the small-world wiring scheme and enable high-speed
			connectivity between a SpiNNaker machine and the outside world.
			
			Continued work on SpiNNaker's interconnect is proposed in chapter
			\ref{sec:research-plan}. This work includes the development of a benchmark
			suite based on a combination of real neural networks and synthetic tests.
			Based on these benchmarks, work will progress with an implementation of
			small-world wiring in SpiNNaker. This is followed by work to construct a
			mechanism to exploit conventional HSS power-management techniques without
			large latency costs would be detrimental to realtime simulations.
